\documentclass{article}
\newcommand\tab[1][1cm]{\hspace*{#1}}

\usepackage{listings}
\usepackage{color}
\usepackage{enumitem}
\usepackage{graphicx}
\graphicspath{ {./images/} }
\setlist{nosep}

\definecolor{dkgreen}{rgb}{0,0.6,0}
\definecolor{gray}{rgb}{0.5,0.5,0.5}
\definecolor{mauve}{rgb}{0.58,0,0.82}

\begin{document}
	\begin{titlepage}
		\begin{center}
			\large
			\textbf{PTRFinance}
			
			\vspace{3cm}
			
			\normalsize
			\textit{Developed and Written by Rocco Pio Maria Petruccio}
		\end{center}
	\end{titlepage}
	
	\tableofcontents
	
	\newpage
	
	\section{Background Information}
		The main reason for having created this library was that of actually creating my first functional Python library, as well as being able to offer people an alternative and possibly better version to current web scraping financial libraries.
		
	\newpage
	
	\section{Function Documentation}
		The following section contains the functions in order as they appear in the PTRFinance library.
		
		\subsection{getLiveValues(stocks)}
			\subsubsection{Purpose}
				The purpose of this function is to allow the user to be able to get the most recent stock values of their desired companies, and make use of that information.
				
			\subsubsection{Parameters}
				The parameters in the function are:
					\begin{itemize}
						\item stocks
							\begin{itemize}
								\item This is used to contain the array with the stocks / stock that the user wants to get the most recent value of.
							\end{itemize}
					\end{itemize}
			
			\subsubsection{Time Complexity}			
				\noindent The function has a time complexity of $O(n)$.
					\begin{itemize}
						\item Given that it has a for loop inside of it.
					\end{itemize}
			
			\subsubsection{Return Type}
				The return type of the function is a dictionary containing the name of the stock as the key, with its specific stock price as its value.
		
		\newpage
		
		\subsection{repetitionsFunc(stockName, interval, repetitions)}
			\subsubsection{Purpose}
				The function is an auxiliary function to the ``whileTrueStock" function.
				
		\newpage
		
		\subsection{stockInformation(url, url1, url2)}
			\subsubsection{Purpose}
				The function is used to gather the basic information of the stock that the user is interested in.
				
			\subsubsection{Parameters}
				The function has the following parameters:
					\begin{itemize}
						\item url
							\begin{itemize}
								\item The first url parameter is used to get the 
							\end{itemize}
					\end{itemize}
			
			\subsubsection{Time Complexity}
			
			\subsubsection{Return Type}
			
		\newpage
		
		\subsection{title}	
			\subsubsection{Purpose}
			
			\subsubsection{Parameters}
			
			\subsubsection{Time Complexity}
			
			\subsubsection{Return Type}
			
		
\end{document}